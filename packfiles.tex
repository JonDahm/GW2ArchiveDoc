\chapter{Pack Files}
\label{chap:packfiles}

This chapter will introduce you to one file type -- The Pack File.  This file
type is used to store a large portion of the game data, and has many subtypes
for data such as animations, models, maps, textures, and audio.  You'll learn
how to navigate through the pack file to grab individual portions of the data
it holds, as well as what data each subtype contains.

\section{Headers}
\label{sec:pfheaders}

Each Pack File begins with a 12-byte header identifying the type of data
contained.  This header includes the 4-byte Character Code identifier labeling
the file as a Pack File. The format of this header can be found in
\autoref{tab:pfheader}.

\begin{datastructure}{htp}{the Pack File header}{pfheader}
	\field{char[2]}{Identifier}{Always [`P',`F']}
	\field{uint32}{Unknown}{Always \hex{01}}
	\field{uint16}{Header Size}{Always 12}
	\field{FourCC}{Type}{4 Character Codes defining the data held in this Pack File.}
\end{datastructure}

Each Pack File is split into blocks of data, called Chunks.  The first chunk
follows immediately after the Pack File header.  Each one contains a pointer
to the next one.

Chunks are identified by 4 character codes, which determines the format of the
chunk and the data stored in it.  While different Pack Files may share chunks
with the same identifier, these chunks may not be the same.  For example, the
Content Manifest and Map Metadata Pack Files both have a chunk labeled 'Main,'
but their formats are different.

Chunks begin with a 16-byte header describing where the next chunk is in the
file and the type of data stored in this chunk.  The format can be found in
\autoref{tab:pfchunkheader}.

\begin{datastructure}{htp}{the Chunk header}{pfchunkheader}
	\field{FourCC}{Identifier}{Type of Chunk}
	\field{uint32}{Next Chunk}{Number of bytes after the end of this field that%
	                           the next chunk appears}
	\field{uint16}{Unknown}{}
	\field{uint16}{Header Size}{Always 12}
	\field{Ptr\tmpl{?}}{Unknown}{Pointer to an unknown data structure}
\end{datastructure}

In \autoref{tab:pftypes} you will find all the known Pack File types, as well
as the chunks each one contains.  The following sections will go into each chunk
and its format.

\begin{table}[htp]\begin{center}
	\caption{Pack File Types sorted by FourCC}
	\label{tab:pftypes}
	
	\begin{tabular}{rrl|p{1.5in}}
		\textbf{FourCC} & \textbf{Value} & \textbf{Name} & \textbf{Included Chunks} \\
		\hline%\fourcc{`0'}{`0'}{`0'}{`0'}
		\fourcc{`A'}{`B'}{`I'}{`X'} & \hex{58494241}
			& \hyperref[sec:pfABIX]{Bank Index} %
			& \fourcc{`B'}{`I'}{`D'}{`X'} \\
		\hline
		\fourcc{`A'}{`B'}{`N'}{`K'} & \hex{4b4e4241}
			& \hyperref[sec:pfABNK]{Bank} %
			& \fourcc{`B'}{`K'}{`C'}{`K'} \\
		\hline
		\fourcc{`A'}{`M'}{`A'}{`T'} & \hex{54414d41}
			& \hyperref[sec:pfAMAT]{Material} %
			& \fourcc{`G'}{`R'}{`M'}{`T'}, \fourcc{`D'}{`X'}{`9'}{`S'} \\
		\hline
		\fourcc{`A'}{`M'}{`S'}{`P'} & \hex{50534d41}
			& \hyperref[sec:pfAMSP]{Script} %
			& \fourcc{`A'}{`M'}{`S'}{`P'} \\
		\hline
		\fourcc{`a'}{`n'}{`i'}{`c'} & \hex{63696E61}
			& \hyperref[sec:pfanic]{Animation Sequence} %
			& \fourcc{`s'}{`e'}{`q'}{`n'} \\
		\hline
		\fourcc{`A'}{`S'}{`N'}{`D'} & \hex{444e5341}
			& \hyperref[sec:pfASND]{Audio} %
			& \fourcc{`A'}{`S'}{`N'}{`D'} \\
		\hline
		\fourcc{`C'}{`I'}{`N'}{`P'} & \hex{504e4943}
			& \hyperref[sec:pfCINP]{Scene} %
			& \fourcc{`C'}{`S'}{`C'}{`N'} \\
		\hline
		\fourcc{`c'}{`m'}{`a'}{`C'} & \hex{43616d63}
			& \hyperref[sec:pfcmaC]{Collide Model Manifest} %
			& \fourcc{`m'}{`a'}{`i'}{`n'} \\
		\hline
		\fourcc{`c'}{`m'}{`p'}{`c'} & \hex{63706d63}
			& \hyperref[sec:pfcmpc]{Composite} %
			& \fourcc{`c'}{`o'}{`m'}{`p'} \\
		\hline
		\fourcc{`c'}{`n'}{`t'}{`c'} & \hex{63746e63}
			& \hyperref[sec:pfcntc]{Content Manifest} %
			& \fourcc{`M'}{`a'}{`i'}{`n'} \\
		\hline
		\fourcc{`e'}{`m'}{`o'}{`c'} & \hex{636f6d65}
			& \hyperref[sec:pfemoc]{Emote Animation} %
			& \fourcc{`a'}{`n'}{`i'}{`m'} \\
		\hline
		\fourcc{`e'}{`u'}{`l'}{`a'} & \hex{616c7565}
			& \hyperref[sec:pfeula]{EULA} %
			& \fourcc{`e'}{`u'}{`l'}{`a'} \\
		\hline
		\fourcc{`h'}{`v'}{`k'}{`C'} & \hex{436b7668}
			& \hyperref[sec:pfhvkC]{Havok} %
			& \fourcc{`h'}{`a'}{`v'}{`k'} \\
		\hline
		\fourcc{`m'}{`a'}{`p'}{`c'} & \hex{6370616d}
			& \hyperref[sec:pfmapc]{Map} %
			& \fourcc{`a'}{`u'}{`d'}{`i'},    \fourcc{`m'}{`s'}{`n'}{\hex{00}},
			\fourcc{`p'}{`a'}{`r'}{`m'},      \fourcc{`s'}{`h'}{`o'}{`r'},
			\fourcc{`s'}{`u'}{`r'}{`f'},      \fourcc{`t'}{`r'}{`n'}{`i'},
			\fourcc{`a'}{`r'}{`e'}{`a'},      \fourcc{`h'}{`a'}{`v'}{`k'},
			\fourcc{`c'}{`u'}{`b'}{`e'},      \fourcc{`d'}{`c'}{`a'}{`l'},
			\fourcc{`e'}{`n'}{`v'}{\hex{00}}, \fourcc{`l'}{`g'}{`h'}{`t'},
			\fourcc{`p'}{`r'}{`p'}{`2'},      \fourcc{`r'}{`i'}{`v'}{`e'},
			\fourcc{`s'}{`h'}{`e'}{`x'},      \fourcc{`t'}{`r'}{`n'}{\hex{00}},
			\fourcc{`z'}{`o'}{`n'}{`2'} \\
		\hline
		\fourcc{`m'}{`M'}{`e'}{`t'} & \hex{74654d6d}
			& \hyperref[sec:pfmMet]{Map Metadata} %
			& \fourcc{`M'}{`a'}{`i'}{`n'} \\
		\hline
		\fourcc{`M'}{`O'}{`D'}{`L'} & \hex{4c444f4d}
			& \hyperref[sec:pfMODL]{Model} %
			& \fourcc{`A'}{`N'}{`I'}{`M'}, \fourcc{`M'}{`O'}{`D'}{`L'},
			\fourcc{`G'}{`E'}{`O'}{`M'}, \fourcc{`P'}{`R'}{`P'}{`S'},
			\fourcc{`R'}{`O'}{`O'}{`T'}, \fourcc{`S'}{`K'}{`E'}{`L'} \\
		\hline
		\fourcc{`m'}{`p'}{`s'}{`d'} & \hex{6473706d}
			& \hyperref[sec:pfmpsd]{Map Shadow} %
			& \fourcc{`s'}{`h'}{`a'}{`d'} \\
		\hline
		\fourcc{`P'}{`I'}{`M'}{`G'} & \hex{474d4950}
			& \hyperref[sec:pfPIMG]{Paged Image} %
			& \fourcc{`P'}{`G'}{`T'}{`B'} \\
		\hline
		\fourcc{`p'}{`r'}{`l'}{`t'} & \hex{746c7270}
			& \hyperref[sec:pfprlt]{Portal Manifest} %
			& \fourcc{`m'}{`f'}{`s'}{`t'} \\
		\hline
		\fourcc{`t'}{`x'}{`t'}{`m'} & \hex{6d747874}
			& \hyperref[sec:pftxtm]{Text Manifest} %
			& \fourcc{`t'}{`x'}{`t'}{`m'} \\
		\hline
		\fourcc{`t'}{`x'}{`t'}{`V'} & \hex{56747874}
			& \hyperref[sec:pftxtV]{Text Variant} %
			& \fourcc{`v'}{`a'}{`r'}{`i'} \\
		\hline
		\fourcc{`t'}{`x'}{`t'}{`v'} & \hex{76747874}
			& \hyperref[sec:pftxtv]{Text Voice} %
			& \fourcc{`t'}{`x'}{`t'}{`v'} \\
		\hline
	\end{tabular}
\end{center}\end{table}

\section{Common Structures}
\label{sec:pfcommonstructures}
Before we go into the different Pack File types and their formats, we should go
over some common structures found in the pack files.  There are three basic
structures and three others that build off of them.

We'll start with the Pointer.  In the documentation, we'll shorten it to Ptr for
brevity.  This structure consists of one 32-bit offset that references data
elsewhere in the file.  The address of the data referenced is the address of the
Ptr plus the offset.

The next basic structure is the Array -- shortened to Arr.  It is simply one
data type T repeated N times.  It will be written as Arr\tmpl{T,N}, T being the
data type and N being the number of data types.  It takes up $size(T)*N$ bytes.

The last basic structure is the File Reference -- shortened to FileRef.  It
occupies 6 bytes, but only 4 of those bytes are useful information.  According
to Johan Sköld, this structure represents a File ID, and it is only valid if the
first two 16-bit values are both above \hex{100}.  In his project, he calculates
the File ID as $first - \hex{FF} + (second - \hex{100}) * \hex{FF00}$.
\emph{More research must be done into whether this equation is correct.}

Moving on to composite structures, the first we'll discuss is the String and
Wide String -- shortened to Str and WStr, respectively.  These are simply Ptrs
to C Strings using either traditional or wide characters, but are common enough
that they warrant their own listing here.

The next composite structure is the Vector -- shortened to Vtr.  This is similar
to an array, but the size is determined at run time and not at compile time, and
it always takes up 8 bytes regardless of the number of elements it represents.
The first 32-bit value in this structure, N, is the number of elements in the
Vector. The second is a Pointer of type Ptr\tmpl{Arr\tmpl{T,N}}, where T is the
data type represented by the Vector.

The last composite structure we'll discuss is the Reference List -- shortened
to RefList.  This is a Vtr of pointers to data types.  It is equivalent to
Vtr\tmpl{Ptr\tmpl{T}}, where T is the data type we're pointing to.

\input{PackFileTypes/anic}
\input{PackFileTypes/ASND}
\section{Bank Pack File}
\label{sec:pfABNK}

This type has only one chunk, whose FourCC is \fourcc{`B'}{`K'}{`C'}{`K'}.  It
has 3 fields, as documented in \autoref{tab:pfABNKchunkBKCK}.


\begin{datastructure}{hpt}{the BKCK chunk format}{pfABNKchunkBKCK}
	\field{byte[16]}{(unknown)}{}
	\field{Vtr\tmpl{Sound}}{Sounds (see \autoref{tab:pfABNKsound})}{}
	\field{Ptr\tmpl{byte[16]}}{(unknown)}{}
\end{datastructure}

\begin{datastructure}{hpt}{the Sound structure format (ABNK)}{pfABNKsound}
	\field{uint32}{Voice ID}{}
	\field{uint32}{Flags}{}
	\field{byte[16]}{(unknown)}{}
	\field{float32}{Length}{}
	\field{float32}{Offset}{}
	\field{byte[16]}{(unknown)}{}
	\field{Vtr\tmpl{byte}}{Audio Data}{}
\end{datastructure}

TODO: Research the purpose of these fields and this file.

\clearpage


\section{Bank Index Pack File}
\label{sec:pfABIX}

TODO: Finish this section




\section{Collide Model Manifest Pack File}
\label{sec:pfcmaC}

TODO: Finish this section




\section{Composite Pack File}
\label{sec:pfcmpc}

TODO: Finish this section

\clearpage


\section{Content Manifest Pack File}
\label{sec:pfcntc}

TODO: Finish this section

\clearpage


\input{PackFileTypes/emoc}
\input{PackFileTypes/eula}
\section{Havok Pack File}
\label{sec:pfhvkC}

TODO: Finish this section




\section{Map Pack File}
\label{sec:pfmapc}

TODO: Finish this section

\subsection{Unknown Chunk}
\label{subsec:pfmapcchunkaudi}

TODO: Finish this section

\subsection{Unknown Chunk}
\label{subsec:pfmapcchunkmsn}

TODO: Finish this section

\subsection{Unknown Chunk}
\label{subsec:pfmapcchunkparm}

TODO: Finish this section

\subsection{Unknown Chunk}
\label{subsec:pfmapcchunkshor}

TODO: Finish this section

\subsection{Unknown Chunk}
\label{subsec:pfmapcchunksurf}

TODO: Finish this section

\subsection{Unknown Chunk}
\label{subsec:pfmapcchunktrni}

TODO: Finish this section

\subsection{Unknown Chunk}
\label{subsec:pfmapcchunkarea}

TODO: Finish this section

\subsection{Unknown Chunk}
\label{subsec:pfmapcchunkhavk}

TODO: Finish this section

\subsection{Unknown Chunk}
\label{subsec:pfmapcchunkcube}

TODO: Finish this section

\subsection{Unknown Chunk}
\label{subsec:pfmapcchunkdcal}

TODO: Finish this section

\subsection{Unknown Chunk}
\label{subsec:pfmapcchunkenv}

TODO: Finish this section

\subsection{Unknown Chunk}
\label{subsec:pfmapcchunklght}

TODO: Finish this section

\subsection{Unknown Chunk}
\label{subsec:pfmapcchunkprp2}

TODO: Finish this section

\subsection{Unknown Chunk}
\label{subsec:pfmapcchunkrive}

TODO: Finish this section

\subsection{Unknown Chunk}
\label{subsec:pfmapcchunkshex}

TODO: Finish this section

\subsection{Unknown Chunk}
\label{subsec:pfmapcchunktrn}

TODO: Finish this section

\subsection{Unknown Chunk}
\label{subsec:pfmapcchunkzon2}

TODO: Finish this section

\clearpage


\input{PackFileTypes/mMet}
\input{PackFileTypes/mpsd}
\section{Material Pack File}
\label{sec:pfAMAT}

The Material Pack File has two chunks -- \fourcc{`G'}{`R'}{`M'}{`T'} and
\fourcc{`D'}{`X'}{`9'}{`S'}.

\subsection{Unknown Chunk (GRMT)}
\label{subsec:pfAMATchunkGRMT}

This chunk has 11 fields:

\begin{datastructure}{h}{GRMT Chunk Format}{pfAMATchunkGRMT}
	\field{byte}{Texture Array Range}{}
	\field{byte}{Texture Count}{}
	\field{byte}{Texture Transformation Range}{}
	\field{byte}{Sort Order}{}
	\field{byte}{Sort Triangles}{}
	\field{byte}{Process Animations}{}
	\field{uint32}{Debug Flags}{}
	\field{uint32}{Flags}{}
	\field{uint32}{Texture Type}{}
	\field{Arr\tmpl{uint32,4}}{Texture Masks}{}
	\field{Vtr\tmpl{uint64}}{Texture Tokens}{}
\end{datastructure}

\subsection{Unknown Chunk (DX9S)}
\label{subsec:pfAMATchunkDX9S}

TODO: Finish this section




\section{Model Pack File}
\label{sec:pfMODL}

TODO: Finish this section

\subsection{Unknown Chunk}
\label{subsec:pfchunkANIM}

TODO: Finish this section

\subsection{Unknown Chunk}
\label{subsec:pfchunkMODL}

TODO: Finish this section

\subsection{Unknown Chunk}
\label{subsec:pfchunkGEOM}

TODO: Finish this section

\subsection{Unknown Chunk}
\label{subsec:pfchunkPRPS}

TODO: Finish this section

\subsection{Unknown Chunk}
\label{subsec:pfchunkROOT}

TODO: Finish this section

\subsection{Unknown Chunk}
\label{subsec:pfchunkSKEL}

TODO: Finish this section

\clearpage


\section{Paged Image Pack File}
\label{sec:pfPIMG}

TODO: Finish this section

\clearpage


\section{Portal Manifest Pack File}
\label{sec:pfprlt}

TODO: Finish this section

\clearpage


\input{PackFileTypes/CINP}
\section{Script Pack File}
\label{sec:pfAMSP}

This Pack File has only one chunk, whose FourCC is \fourcc{`A'}{`M'}{`S'}{`P'}.
However, there is a lot of data in that single chunk, with 22 different
structures included in it.  See \autoref{tab:pfAMSPchunkAMSP} for the chunk's
format.

\begin{datastructure}{hp}{the AMSP chunk format.}{pfAMSPchunkAMSP}
	\field{uint64}{Music Cue}{}
	\field{uint64}{Reverb Override}{}
	\field{uint64}{Snapshot}{}
	\field{Ptr\tmpl{AudioSettings}}{Audio Settings}{(see \autoref{tab:pfAMSPaudiosettings})}
	\field{Vtr\tmpl{Handler}}{Handlers}{(see \autoref{tab:pfAMSPhandler})}
	\field{Vtr\tmpl{MetaSoundData}}{Meta Sound Data}{(see \autoref{tab:pfAMSPmetasounddata})}
	\field{Vtr\tmpl{ScriptRef}}{Script References}{(see \autoref{tab:pfAMSPscriptref})}
	\field{Vtr\tmpl{TriggerKey}}{Trigger Keys}{(see \autoref{tab:pfAMSPtriggerkey})}
	\field{uint32}{Flags}{}
	\field{uint32}{Sound Pool Count}{}
	\field{float32}{Fade-in Time}{}
	\field{float32}{Sound Pool Delay}{}
	\field{float32}{Volume}{}
	\field{byte}{Music Cue Priority}{}
	\field{byte}{Music Mute Priority}{}
\end{datastructure}

%\clearpage

\begin{datastructure}{p}{the AudioSettings structure}{pfAMSPaudiosettings}
	\field{uint64}{Default Snapshot}{}
	\field{uint64}{Effects Bus}{}
	\field{float32}{Distance Scale}{}
	\field{float32}{Doppler Scale}{}
	\field{float32}{Focus Transition}{}
	\field{Vtr\tmpl{BusData}}{Busses}{(see \autoref{tab:pfAMSPbusdata})}
	\field{Vtr\tmpl{Category}}{Categories}{(see \autoref{tab:pfAMSPcategory})}
	\field{Vtr\tmpl{MusicCondition}}{Music Conditions}{(see \autoref{tab:pfAMSPmusiccondition})}
	\field{Vtr\tmpl{Playlist}}{Playlists}{(see \autoref{tab:pfAMSPplaylist})}
	\field{Vtr\tmpl{Reverb}}{Reverbs}{(see \autoref{tab:pfAMSPreverb})}
	\field{Vtr\tmpl{Snapshot}}{Snapshots}{(see \autoref{tab:pfAMSPsnapshot})}
	\field{Ptr\tmpl{FileRef}}{Bank Index File}{}
	\field{Ptr\tmpl{FileRef}}{Bank Script File}{}
	\field{Ptr\tmpl{FileRef}}{Music Script File}{}
\end{datastructure}

\begin{datastructure}{p}{the BusData structure}{pfAMSPbusdata}
	\field{uint64}{ID}{}
	\field{uint32}{Flags}{}
	\field{uint64}{Output}{}
	\field{Ptr\tmpl{BusDynamicData}}{Dynamic Data}{(see \autoref{tab:pfAMSPbusdynamicdata})}
\end{datastructure}

%\clearpage

\begin{datastructure}{p}{the BusDynamicData structure}{pfAMSPbusdynamicdata}
	\field{uint64}{ID}{}
	\field{uint32}{Flags}{}
	\field{float32}{Volume}{}
	\field{Vtr\tmpl{DSPData}}{DSP Data}{(see \autoref{tab:pfAMSPdspdata})}
\end{datastructure}

\begin{datastructure}{p}{the DSPData structure}{pfAMSPdspdata}
	\field{uint32}{Type}{}
	\field{uint32}{Flags}{}
	\field{Vtr\tmpl{float32}}{Properties}{}
\end{datastructure}

\begin{datastructure}{p}{the Category structure}{pfAMSPcategory}
	\field{uint64}{ID}{}
	\field{uint64}{Volume Group ID}{}
	\field{uint64}{Output Bus ID}{}
	\field{Ptr\tmpl{Attenuation}}{Attenuation}{(see \autoref{tab:pfAMSPattenuation})}
	\field{Ptr\tmpl{CategoryDynamicData}}{Dynamic Data}{(see \autoref{tab:pfAMSPcategorydynamicdata})}
	\field{float32}{Mute Fade Time}{}
	\field{uint32}{Flags}{}
	\field{uint32}{Max Audible}{}
	\field{byte}{Max Audible Behavior}{}
\end{datastructure}

%\clearpage

\begin{datastructure}{p}{the Attenuation structure}{pfAMSPattenuation}
	\field{float32}{Doppler}{}
	\field{DynamicParamData}{Low Pass}{(see \autoref{tab:pfAMSPdynamicparamdata})}
	\field{DynamicParamData}{High Pass}{}
	\field{DynamicParamData}{Pan 3D}{}
	\field{DynamicParamData}{Reverb}{}
	\field{DynamicParamData}{Spread 3D}{}
	\field{DynamicParamData}{Volume A}{}
	\field{DynamicParamData}{Volume B}{}
\end{datastructure}

\begin{datastructure}{p}{the DynamicParamData structure}{pfAMSPdynamicparamdata}
	\field{Ptr\tmpl{Envelope}}{Envelope}{(see \autoref{tab:pfAMSPenvelope})}
	\field{Ptr\tmpl{RandomParam}}{Random Param}{(see \autoref{tab:pfAMSPrandomparam})}
	\field{float32}{Value}{}
	\field{byte}{Type}{}
\end{datastructure}

\begin{datastructure}{p}{the Envelope structure}{pfAMSPenvelope}
	\field{uint64}{Offset Parameter}{}
	\field{Vtr\tmpl{EnvelopePoint}}{Envelope Points}{(see \autoref{tab:pfAMSPenvelopepoint})}
	\field{byte}{Offset Type}{}
\end{datastructure}

\begin{datastructure}{p}{the EnvelopePoint structure}{pfAMSPenvelopepoint}
	\field{float32}{Offset}{}
	\field{float32}{Value}{}
\end{datastructure}

%\clearpage

\begin{datastructure}{p}{the RandomParam structure}{pfAMSPrandomparam}
	\field{RangeData}{Time}{(see \autoref{tab:pfAMSPrangedata})}
	\field{RangeData}{Value}{}
\end{datastructure}

\begin{datastructure}{p}{the RangeData structure}{pfAMSPrangedata}
	\field{float32}{Max}{}
	\field{float32}{Min}{}
	\field{byte}{Min (byte)}{}
\end{datastructure}

\begin{datastructure}{p}{the CategoryDynamicData structure}{pfAMSPcategorydynamicdata}
	\field{uint64}{ID}{}
	\field{float32}{Volume}{}
	\field{float32}{Non-Focus Gain}{}
	\field{float32}{Low Pass}{}
	\field{float32}{High Pass}{}
	\field{float32}{Reverb Direction}{}
	\field{float32}{Reverb Room}{}
	\field{uint32}{Flags}{}
\end{datastructure}

\begin{datastructure}{p}{the MusicCondition structure}{pfAMSPmusiccondition}
	\field{uint64}{ID}{}
	\field{uint32}{Flag}{}
	\field{Vtr\tmpl{byte}}{Byte Code}{}
\end{datastructure}

%\clearpage

\begin{datastructure}{p}{the Playlist structure}{pfAMSPplaylist}
	\field{uint64}{Category}{}
	\field{uint64}{ID}{}
	\field{uint64}{Primary Playlist ID}{}
	\field{uint64}{Secondary Playlist ID}{}
	\field{Vtr\tmpl{FileNameData}}{Files}{(see \autoref{tab:pfAMSPfilenamedata})}
	\field{float32}{Fade-In Time}{}
	\field{float32}{Fade-Out Time}{}
	\field{uint32}{Flags}{}
	\field{RangeData}{Initial Silence}{}
	\field{RangeData}{Interval Silence}{}
	\field{RangeData}{Max Play Length}{}
	\field{DynamicParamData}{Volume}{}
	\field{byte}{File Iterate Mode}{}
\end{datastructure}

\begin{datastructure}{p}{the FileNameData structure}{pfAMSPfilenamedata}
	\field{uint64}{Condition}{}
	\field{uint64}{Language}{}
	\field{float32}{Volume}{}
	\field{float32}{Weight}{}
	\field{Ptr\tmpl{FileRef}}{File}{}
	\field{byte}{Audio Type}{}
	\field{byte}{Note Base}{}
	\field{byte}{Note Min}{}
	\field{byte}{Note Max}{}
\end{datastructure}

%\clearpage

\begin{datastructure}{p}{the Reverb structure}{pfAMSPreverb}
	\field{uint64}{ID}{}
	\field{uint32}{Flags}{}
	\field{float32}{Room}{}
	\field{float32}{Room HF}{}
	\field{float32}{Room LF}{}
	\field{float32}{Decay Time}{}
	\field{float32}{Decay HF Ratio}{}
	\field{float32}{Reflections}{}
	\field{float32}{Reflections Delay}{}
	\field{float32}{Reverb}{}
	\field{float32}{Reverb Delay}{}
	\field{float32}{Reference HF}{}
	\field{float32}{Reference LF}{}
	\field{float32}{Diffusion}{}
	\field{float32}{Density}{}
	\field{float32}{Echo Delay}{}
	\field{float32}{Echo Decay Ration}{}
	\field{float32}{Echo Wet Mix}{}
	\field{float32}{Echo Dry Mix}{}
\end{datastructure}

\clearpage

\begin{datastructure}{p}{the Snapshot structure}{pfAMSPsnapshot}
	\field{uint64}{ID}{}
	\field{float32}{Blend-In Time}{}
	\field{float32}{Blend-Out Time}{}
	\field{uint32}{Flags}{}
	\field{Vtr\tmpl{BusDynamicData}}{Busses}{(see \autoref{tab:pfAMSPbusdynamicdata})}
	\field{Vtr\tmpl{CategoryDynamicData}}{Categories}{(see \autoref{tab:pfAMSPcategorydynamicdata})}
	\field{byte}{Priority}{}
\end{datastructure}

\begin{datastructure}{p}{the Handler structure}{pfAMSPhandler}
	\field{uint64}{ID}{}
	\field{uint32}{Flags}{}
	\field{Vtr\tmpl{byte}}{Byte Code}{}
\end{datastructure}

\begin{datastructure}{!p}{the MetaSoundData structure}{pfAMSPmetasounddata}
	\field{uint64}{Category}{}
	\field{uint64}{End Cue}{}
	\field{uint64}{ID}{}
	\field{uint64}{Offset Bone}{}
	\field{uint64}{Playlist ID}{}
	\field{Vtr\tmpl{DSPData}}{DSP}{(see \autoref{tab:pfAMSPdspdata})}
	\field{Ptr\tmpl{Attenuation}}{Attenuation}{(see \autoref{tab:pfAMSPattenuation})}
	\field{Vtr\tmpl{FileNameData}}{Files}{(see \autoref{tab:pfAMSPfilenamedata})}
	\field{float32}{Channel Fade-In}{}
	\field{float32}{Channel Fade-Out}{}
	\field{float32}{End Cue Offset}{}
	\field{float32}{Fade-In Time}{}
	\field{float32}{Fade-Out Time}{}
	\field{float32[3]}{Position Offset}{}
	\field{uint32}{Channel Max}{}
	\field{uint32}{Flags}{}
	\field{uint32}{Loop Count}{}
	\field{DynamicParamData}{Depth}{(see \autoref{tab:pfAMSPdynamicparamdata})}
	\field{DynamicParamData}{Pan}{}
	\field{DynamicParamData}{Pitch}{}
	\field{DynamicParamData}{Pitch MS}{}
	\field{DynamicParamData}{Volume}{}
	\field{DynamicParamData}{Volume MS}{}
	\field{RangeData}{Initial Delay}{(see \autoref{tab:pfAMSPrangedata})}
	\field{RangeData}{Play Length}{}
	\field{RangeData}{Position Offset Angle}{}
	\field{RangeData}{Position Range}{}
	\field{RangeData}{Repeat Count}{}
	\field{RangeData}{Repeat Time}{}
	\field{RangeData}{Start Time Offset}{}
	\field{byte}{Channel Mode}{}
	\field{byte}{Channel Priority}{}
	\field{byte}{File Iterate Mode}{}
	\field{byte}{Loop Mode}{}
	\field{byte}{Music Priority}{}
	\field{byte}{Playback Mode}{}
	\field{byte}{Position Mode}{}
	\field{byte}{Repeat Time From}{}
\end{datastructure}

\begin{datastructure}{htbp}{the ScriptRef structure}{pfAMSPscriptref}
	\field{uint64}{ID}{}
	\field{Ptr\tmpl{FileRef}}{File}{}
\end{datastructure}

\begin{datastructure}{p}{the TriggerKey structure}{pfAMSPtriggerkey}
	\field{uint64}{ID}{}
	\field{Vtr\tmpl{TriggerMarker}}{Markers}{(see \autoref{tab:pfAMSPtriggermarker})}
\end{datastructure}

\begin{datastructure}{p}{the TriggerMarker structure}{pfAMSPtriggermarker}
	\field{uint64}{Cue}{}
	\field{uint64}{End}{}
	\field{float32}{Time}{}
	\field{byte}{Type}{}
\end{datastructure}

\clearpage


\section{Text Manifest Pack File}
\label{sec:pftxtm}

TODO: Finish this section

\clearpage


\section{Text Variant Pack File}
\label{sec:pftxtV}

TODO: Finish this section

\clearpage

\section{Text Voice Pack File}
\label{sec:pftxtv}

TODO: Finish this section

\clearpage



