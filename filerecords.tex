\section{File Records}
\label{sec:filerecords}

This chapter will introduce you to the main portions of the Archive, from which 
you can find every file represented within.  After reading this chapter, you
should be able to produce a list of all files within the archive.  Additionally,
if a file within the archive is referenced by its ID, you should be able to
retrieve it.

\subsection{The Archive Header}
\label{subsec:archhead}

The Archive begins with a 40-byte header which describes some of the properties
of the Archive and points to the Main File Table.  The format of this header
can be found in Table \ref{tab:archhead}.

\begin{table}[htp]\begin{center}
	\caption{the Archive header}
	\label{tab:archhead}
	
	\begin{tabular}{|r|r|l|p{2.5in}|}
		\hline
		\textbf{Byte} & \textbf{Size} & \textbf{Value} & \textbf{Description} \\
		\hline
		 0 & 1 & Version     & Version of the Archive. Seems to always be \hex{97}  \\
		\hline
		 1 & 3 & Identifier  & Identifies this file as the Archive file, as
		                       opposed to a MS Word file.  Always
							   [\hex{45},\hex{4E},\hex{1A}].  \\
		\hline
		 4 & 4 & Header Size & Size of this header.  Always 40.  \\
		\hline
		 8 & 4 & (unknown)   & Always \hex{CABA0001}.  \\
		\hline
		12 & 4 & Chunk Size  & Size of each chunk in the file.  Always 512.  \\
		\hline
		16 & 4 & (unknown)\footnote{In the GW1 Archive, this was the CRC-32 value
									of the first 16 bytes of the header.} %
							 & Always \hex{8ED0A720}.  \\
		\hline
		20 & 4 & (unknown)   & Always \hex{00040002}.  \\
		\hline
		24 & 8 & MFT Offset  & The offset from the beginning of the Archive to
		                       the Main File Table.  \\
		\hline
		32 & 4 & MFT Size    & Size of the Main File Table in bytes.  \\
		\hline
		36 & 4 & (unknown)   & Always 0. \\
		\hline
	\end{tabular}
\end{center}\end{table}

\subsection{The Main File Table}
\label{subsec:mft}

The Main File Table (MFT) is a list of all of the files in the Archive.  Its
structure begins with a 24-byte-long header, whose format is given in Table
\ref{tab:mfthead}.  The header is followed by a number of 24-byte entries that
make up the table.  Each entry refers to a single file and some associated
metadata.  The entries are not listed in any particular order.  See Table
\ref{tab:mftent} for details.

The first fifteen entries in the MFT are reserved for special files in the
Archive.  They are documented below:
\\

\begin{tabular}{rl}
	\hline
	1     & Archive Header  \\
	2     & File ID Table (See Section \ref{subsec:fidtab}) \\
	3     & MFT (self reference)  \\
	4--15 & Blank Entries  \\
	\hline
\end{tabular}

\begin{table}[htbp]\begin{center}
	\caption{the MFT header}
	\label{tab:mfthead}
	
	\begin{tabular}{|r|r|l|p{2.5in}|}
		\hline
		\textbf{Byte} & \textbf{Size} & \textbf{Value} & \textbf{Description} \\
		\hline
		 0 & 4 & Identifier & Identifies the start of the MFT.  Always
		                      \fourcc{`M'}{`f'}{`t'}{\hex{1A}}.  \\
		\hline
		 4 & 8 & (unknown)  & \\
		\hline
		12 & 4 & Length     & Number of entries in the table plus one.  \\
		\hline
		16 & 8 & (unknown)  & Always 0.  \\
		\hline
	\end{tabular}
\end{center}\end{table}

\begin{table}[htbp]\begin{center}
	\caption{an MFT entry}
	\label{tab:mftent}
	
	\begin{tabular}{|r|r|l|p{2.5in}|}
		\hline
		\textbf{Byte} & \textbf{Size} & \textbf{Value} & \textbf{Description} \\
		\hline
		 0 & 8 & Offset        & Offset from the beginning of the Archive to the
		                         start of the file.  \\
		\hline
		 8 & 4 & Archived Size & Size in bytes of the file within the archive.  \\
		\hline
		12 & 2 & Compression   & Type of compression the file is under.  See below.  \\
		\hline
		14 & 2 & Flags         & Other flags.  See below.  \\
		\hline
		16 & 4 & (unknown)     & Always 0.  \\
		\hline
		20 & 4 & (unknown)     & Always \hex{4867~4BC7}.  \\
		\hline
	\end{tabular}
	
	Valid values for Compression:
	\begin{tabular}{rl}
		\hline
		0 & Uncompressed \\
		8 & Huffman Compression \\
		\hline
	\end{tabular}
	\\
	
	Valid values for Flags:
	\begin{tabular}{rl}
		\hline
		1 & In Use \\
		2 & (unknown) \\
		\hline
	\end{tabular}
\end{center}\end{table}


\subsection{The File ID Table}
\label{subsec:fidtab}

The File ID Table gives each file in the MFT an ID.  Each entry in the table
has the format listed in table \ref{tab:fident}.  The entries are not listed in
any particular order.

For the most part, each entry has only one ID.  However, many have more than one
ID each.  As of the time of this writing, approximately a third of the files in
the Archive have two IDs, and none have more.  \emph{More research must be done
into why some entries have multiple IDs.}

Additionally, some entries may contain nil values for either field.  I haven't
found a significant number of these, but they exist.  I have only found entries
where both fields are nil, and none where only one was nil.  My recommendation
is to discard any entries with nil fields.

\begin{table}[htbp]\begin{center}
	\caption{a File ID Table entry}
	\label{tab:fident}

	\begin{tabular}{|r|r|l|p{2in}|}
		\hline
		\textbf{Byte} & \textbf{Size} & \textbf{Value} & \textbf{Description}
		\\\hline
		0 & 4 & File ID         &
		\\\hline
		4 & 4 & MFT Entry Index & Indices start at 1
		\\\hline
	\end{tabular}
\end{center}\end{table}

