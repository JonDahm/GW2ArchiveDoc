\section*{Notes}
\label{sec:notes}

\subsection*{Libraries}
\label{subsec:libs}

To my knowledge, there are two major C++ libraries for working with the Archive
file.  Github user Ahom has created a library for working with File Records and
extracting images that you can find 
\href{https://github.com/ahom/gw2DatTools/}{here}.  Github user Rhoot has
created a library that will extract information from a large number of files
within the Archive.  You can find his work
\href{https://github.com/rhoot/gw2formats}{here}.
Most of the information in this document has come from these projects.

\subsection*{Endianness and Numbers}
\label{subsec:endiannum}

All numbers I list in this document are decimal (base 10) unless specified
otherwise.  Hexadecimal numbers are followed by a subscript x (\hex{1A}).
Sometimes a single byte will be listed as a character rather than a number.  In
these cases the value of that byte is the ASCII code of the character listed.

When I list values, sometimes I will list them as full numbers (like \hex{40CB})
and sometimes I will list them as individual bytes (like [\hex{CB},\hex{40}]).
When I list the individual bytes, they are listed in the order they appear in
the Archive.  When I list them as full numbers, that is their actual value.

The Archive is arranged in little-endian format.  This means that if you see a
16-bit value [\hex{CB},\hex{40}], its actual value is \hex{40CB}.

\subsection*{Disclaimer}
\label{subsec:disclaimer}

I do not condone use of this document to modify the archive for any reason.
Modifying the archive is a direct violation of the Terms of Service you agreed
to follow when you bought the game.


