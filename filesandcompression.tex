\section{Files and Compression}
\label{sec:filesandcompression}

This chapter will introduce you to how to identify files and decompress files
that have been compressed.  Additionally, I'll discuss the compression used on
many of the texture files in the Archive.  After reading this chapter, you
should be able to, given the address of the start of a file, provide its raw
data, whether the file was compressed or not.

\subsection{File Types}
\label{subsec:filetypes}

Every file starts with an 8-byte header identifying the type of file and how
large it is.  The first 4 bytes of the header are the file's type identifier,
typically represented by four character codes (4CC).  The second 4 bytes tell
you how long the uncompressed file is, if the file is compressed.

In the latest version of the Archive at the time of this writing, 99\% of the
files were compressed.  All of these files are represented in the general file
header by one 4CC.  To find the actual 4CC defining the file type, you have to
decompress the file, which we will go over in the next section.

The following table describes all 4CCs that appear in the general file header,
listed in decreasing order of frequency:
\\

\begin{tabular}{rl}
	\hline
	\fourcc{\hex{08}}{\hex{00}}{\hex{01}}{\hex{80}} & Compressed File  \\
	\fourcc{`A'}{`T'}{`E'}{`X'} & General Use Texture  \\
	\fourcc{`A'}{`T'}{`E'}{`U'} & UI Texture  \\
	\fourcc{`K'}{`B'}{`2'}{`f'} & (unknown)  \\
	\fourcc{`K'}{`B'}{`2'}{`g'} & (unknown)  \\
	\fourcc{\hex{7C}}{\hex{1A}}{`I'}{`z'} & (unknown)  \\
	\fourcc{\hex{97}}{`A'}{`N'}{\hex{1A}} & (unknown)  \\
	\hline
\end{tabular}

\subsection{Compression}
\label{subsec:compression}

Note to self: Add illustrations.

Compression is a difficult subject to describe tersely.  The compression used in
the Archive is very similar to that produced by the DEFLATE algorithm.  If you
are familiar with the DEFLATE algorithm, you may notice them.  To keep things
(relatively) short, however, I won't describe every difference between the two.

Data is compressed using Huffman codes and back-copying.  The former is a method
of taking a set of data and compressing it as small as possible, and the latter
is a method of further compressing the data by replacing reoccuring data with a
refrence to the last time it occured.  I won't go into the details of how all
this works, so if you aren't familiar with either of these, read
\href{http://zlib.net/feldspar.html}{this fantastic article on zlib} which does
a wonderful job explaining the concepts.  Be sure to understand these concepts
well before continuing in this section, or you will be lost.  If this is well
beyond you, and you don't care particularly about implementing a decompression
algorithm yourself, just use Ahom's decompression algorithm and skip the rest
of this chapter.

To begin, it is incredibly important to note the order in which bits are read.
Strangely enough, bytes aren't read from beginning to end --- instead, they
are split into little-endian 32-bit values, and read from highest bit to lowest.
For illustration, see Figure BLAH.  When I refer to ordering of elements in this
section, I assume that bits are being read in this order.

Next, every 64KiB, 4 bytes are skipped.  As of the writing of this document, I
am unaware the purpose of this.  I would guess that those 4 bytes are a check
on the previous data in order to help detect corruption.

The compressed data starts with a single byte that represents an adjustment to
any back-copy sizes encountered in the data.  This should be saved for later
use.  The rest of the data is split into blocks.

Each block begins with two Huffman Trees describing the Huffman codes for the
literal/copy-length alphabet and the copy-offset alphabet. These are followed
by 4 bits which represent the number of codes from the first alphabet to expect
in this section. The rest of the block is the Huffman codes representing the
information compressed in this section.

In the next subsection, I'll describe how you generate Huffman codes from the
Huffman Trees presented in each block.

\subsection{Huffman Trees}
\label{subsec:huffmantrees}

Each tree can represent a variable number of values.  The first 16 bits are an
unsigned value representing how many values this tree is giving Huffman codes
to.  This is followed by a number of entries describing sometimes several codes
at once.  These entries are compressed using predefined codes found in Appendix
A.

Each entry represents at least one value and its code.  The first entry refers
to the highest values the tree represents, with each successive entry referring
to a lower value.  The highest three bits state how many more values this entry
applies to.  The lowest 5 bits state how long the Huffman codes are for these
values.  If the length is 0, those values aren't actually represented in the
tree, and you can skip over them.

As it turns out, in order to generate a valid Huffman code for a value, all you
need to know is how long the Huffman code for it is.  The following algorithm
derives the Huffman codes for all values whose lengths you know are non-zero.

Sort all of your value+code-length pairs first in ascending order of length,
then in ascending order of value.  Assign the first value a code of all 1's.
For each successive value that uses the same length code, decrement the code by
one.  When you reach a value that uses more code bits, multiply the last code by
2 and then subtract one.  Continue this process until you have assigned each
value a Huffman code.

The Tree representing the literal/copy-length alphabet cannot have more than
285 values in it. The Tree representing the copy-offset alphabet cannot have
more than 34 values.

\subsection{Translating Huffman Codes to Data}
\label{subsec:codetodata}

In each block, after the Huffman Trees, there are 4 bits describing how many
codes from the literal/copy-length alphabet there in the block.  The number is
determined by adding one to the value of the 4 bits and then multiplying by
\hex{1000}.  If the end of the file has been reached, then this number may be
greater than the actual number of codes, so you'll have to watch to make sure
you don't overshoot the end of the stream.

There are two modes to translating the codes to data --- literal, where each
code matches one byte, and copy, where extra data follows the code describing
how many bytes to copy from where in the output stream generated so far.  If the
value of the code translated is less than \hex{100}, then the output is a byte
with that value.  If the value is greater than \hex{100}, then you have to copy
previous output back into the stream.

Following a copy code are additional bits that add to the length represented by
the code itself.  Table \ref{tab:copylength} provides the base lengths for each
value and how many additional bits you must read and add to the base length.

\begin{table}[htp]\begin{center}
	\caption{Copy Length Table}
	\label{tab:copylength}
	
	\begin{tabular}{|r|r|r|c|r|r|r|}
		\hline
		\textbf{Code} & \textbf{Base} & \textbf{Additional Bits} & & %
		\textbf{Code} & \textbf{Base} & \textbf{Additional Bits} \\
		\hline
		\hex{100} &  1 & 0 & & \hex{110} &  33 & 3 \\
		\hex{101} &  2 & 0 & & \hex{111} &  41 & 3 \\
		\hex{102} &  3 & 0 & & \hex{112} &  49 & 3 \\
		\hex{103} &  4 & 0 & & \hex{113} &  57 & 3 \\
		\hline
		\hex{104} &  5 & 0 & & \hex{114} &  65 & 4 \\
		\hex{105} &  6 & 0 & & \hex{115} &  81 & 4 \\
		\hex{106} &  7 & 0 & & \hex{116} &  97 & 4 \\
		\hex{107} &  8 & 0 & & \hex{117} & 113 & 4 \\
		\hline
		\hex{108} &  9 & 1 & & \hex{118} & 129 & 5 \\
		\hex{109} & 11 & 1 & & \hex{119} & 161 & 5 \\
		\hex{10A} & 13 & 1 & & \hex{11A} & 193 & 5 \\
		\hex{10B} & 15 & 1 & & \hex{11B} & 225 & 5 \\
		\hline
		\hex{10C} & 17 & 2 & & \hex{11C} & 256 & 0 \\
		\hex{10D} & 21 & 2 & & & & \\
		\hex{10E} & 25 & 2 & & & & \\
		\hex{10F} & 29 & 2 & & & & \\
		\hline
	\end{tabular}
\end{center}\end{table}

After that is a code from the copy-offset alphabet.  This also has additional
bits following it to add to it. Table \ref{tab:copyoffset} details the base
offsets and the number of additional bits for each value.

\begin{table}[htp]\begin{center}
	\caption{Copy Offset Table}
	\label{tab:copyoffset}
	
	\begin{tabular}{|r|r|r|c|r|r|r|}
		\hline
		\textbf{Code} & \textbf{Base} & \textbf{Additional Bits} & & %
		\textbf{Code} & \textbf{Base} & \textbf{Additional Bits} \\
		\hline
		\hex{0}  &   \hex{1} & 0 & & \hex{12} &   \hex{201} &  8 \\
		\hex{1}  &   \hex{2} & 0 & & \hex{13} &   \hex{301} &  8 \\
		\hex{2}  &   \hex{3} & 0 & & \hex{14} &   \hex{401} &  9 \\
		\hex{3}  &   \hex{4} & 0 & & \hex{15} &   \hex{601} &  9 \\
		\hex{4}  &   \hex{5} & 1 & & \hex{16} &   \hex{801} & 10 \\
		\hex{5}  &   \hex{7} & 1 & & \hex{17} &   \hex{C01} & 10 \\
		\hex{6}  &   \hex{9} & 2 & & \hex{18} &  \hex{1001} & 11 \\
		\hex{7}  &   \hex{D} & 2 & & \hex{19} &  \hex{1801} & 11 \\
		\hex{8}  &  \hex{11} & 3 & & \hex{1A} &  \hex{2001} & 12 \\
		\hex{9}  &  \hex{19} & 3 & & \hex{1B} &  \hex{3001} & 12 \\
		\hex{A}  &  \hex{21} & 4 & & \hex{1C} &  \hex{4001} & 13 \\
		\hex{B}  &  \hex{31} & 4 & & \hex{1D} &  \hex{6001} & 13 \\
		\hex{C}  &  \hex{41} & 5 & & \hex{1E} &  \hex{8001} & 14 \\
		\hex{D}  &  \hex{61} & 5 & & \hex{1F} &  \hex{C001} & 14 \\
		\hex{E}  &  \hex{81} & 6 & & \hex{20} & \hex{10001} & 15 \\
		\hex{F}  &  \hex{C1} & 6 & & \hex{21} & \hex{18001} & 15 \\
		\hex{10} & \hex{101} & 7 & & & & \\
		\hex{11} & \hex{181} & 7 & & & & \\
		\hline
	\end{tabular}
\end{center}\end{table}

To calculate the total length of the copy, add the base length, the value of the
additional length bits, and the copy size adjustment value from the beginning of
the file.  To calculate the total offset of the copy, add the base offset and
the additional offset bits.  It may be helpful to note that the sliding window
on this algorithm appears to be 128KiB.

