\section{Files and Compression}
\label{sec:filesandcompression}

This chapter will introduce you to how to identify files and decompress files
that have been compressed.  Additionally, I'll discuss the compression used on
many of the texture files in the Archive.  After reading this chapter, you
should be able to, given the address of the start of a file, provide its raw
data, whether the file was compressed or not.

\subsection{File Types}
\label{subsec:filetypes}

Every file starts with an 8-byte header identifying the type of file and how
large it is.  The first 4 bytes of the header are the file's type identifier,
typically represented by four character codes (4CC).  The second 4 bytes tell
you how long the uncompressed file is, if the file is compressed.

In the latest version of the Archive at the time of this writing, 99\% of the
files were compressed.  All of these files are represented in the general file
header by one 4CC.  To find the actual 4CC defining the file type, you have to
decompress the file, which we will go over in the next section.

The following table describes all 4CCs that appear in the general file header,
listed in decreasing order of frequency:
\\

\begin{tabular}{rl}
	\hline
	\fourcc{\hex{08}}{\hex{00}}{\hex{01}}{\hex{80}} & Compressed File  \\
	\fourcc{`A'}{`T'}{`E'}{`X'} & General Use Texture  \\
	\fourcc{`A'}{`T'}{`E'}{`U'} & UI Texture  \\
	\fourcc{`K'}{`B'}{`2'}{`f'} & (unknown)  \\
	\fourcc{`K'}{`B'}{`2'}{`g'} & (unknown)  \\
	\fourcc{\hex{7C}}{\hex{1A}}{`I'}{`z'} & (unknown)  \\
	\fourcc{\hex{97}}{`A'}{`N'}{\hex{1A}} & (unknown)  \\
	\hline
\end{tabular}



